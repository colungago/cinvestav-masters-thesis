% !TEX root = ../my-thesis.tex
%
\pdfbookmark[0]{Resumen}{Resumen}
\addchap*{Resumen}

Esta tesis investiga el impacto de la variación reportada en la conductividad de los tejidos que componen la cabeza en la localización de fuentes de actividad neuronal, particularmente la razón de los valores conductividad del cerebro y cráneo (BSCR).
El estudio se centra en los problemas directo e inverso de la electroencefalografía (EEG), y realiza un análisis estadístico del error incurrido utilizando diferentes valores nominales del BSCR.
La propuesta que se maneja en esta tesis es la de utilizar la frontera de Crámer-Rao (CRB) como parámetro de análisis, esto en función de que la variabilidad inducida por la incertidumbre en el BSCR está acotada por dicha frontera.
La metodología incluye simulaciones de la actividad neuronal utilizando un modelo geométrico realista y el método de elementos de frontera (BEM), realizadas con diferentes valores de BSCR y niveles de relación señal-ruido (SNR).
Los resultados destacan el impacto de las variaciones de conductividad en la precisión de la localización de fuentes, proporcionando información para mejorar las herramientas de diagnóstico e investigación basadas en EEG.





\addchap*{Abstract}

This thesis investigates the impact of reported variation in the conductivity of tissues that compose the head on the localization of sources of neuronal activity, particularly the brain-skull-conductivity-ratio (BSCR). 
The study focuses on the direct and inverse problems of electroencephalography (EEG) and performs a statistical analysis of the error incurred using different nominal BSCR values.
This thesis proposes the use of the Cramér-Rao bound (CRB) as the parameter of analysis. This is possible because the error induced by the uncertainty in the BSCR values asymptotically approaches the CRB.
The methodology includes simulations of neuronal activity using a realistic geometric model and the boundary element method (BEM), performed with different BSCR values and levels of signal-to-noise ratio (SNR). 
The results highlight the impact of conductivity variations on the accuracy of source localization, providing information to improve EEG-based diagnostic and research tools.

