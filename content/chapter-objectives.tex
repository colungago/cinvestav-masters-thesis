% !TEX root = ../my-thesis.tex
%
\chapter{Hipótesis y Objetivos}
\label{sec:obj}

\section{Hipótesis}
\label{sec:obj:hipotesis}

Existen rangos de error tolerables al definir una razón de conductividad eléctrica cerebro/cráneo en la solución del problema inverso en EEG y la tolerancia estará dictada por la frontera de Cramér-Rao.

\section{Objetivo Principal}
\label{sec:obj:main}

Implementar un método de estimación del valor de la razón entre la conductividad eléctrica cerebro/cráneo basado en el cálculo de modelos de electroencefalograma (EEG) en geometrías realistas obtenidas con el método de elementos de frontera (BEM del inglés \emph{Boundary element method}).

\section{Objetivos Particulares}
\label{sec:obj:individual}

\begin{itemize}
	\item Implementar el método del cálculo de EEG rápidamente recalculable propuesto por \citeauthor{Ermer2001}\cite{Ermer2001} para el caso de una fuente de actividad cerebral conocida y valores típicos del BSCR.
	\item Calcular el costo computacional de dicha implementación para una serie de valores discretos del BSCR.
	\item Probar la aplicabilidad del método propuesto en una prueba piloto para el caso de datos reales de EEG de una respuesta sensorial evocada MODIFICAR O REMOVER.
\end{itemize}