% !TEX root = ../my-thesis.tex
%
\chapter{Hipótesis y Objetivos}
\label{sec:obj}

\section{Hipótesis}
\label{sec:obj:hipotesis}

Existen rangos de error tolerables al definir una razón de conductividad eléctrica cerebro/cráneo en la solución del problema inverso en EEG y la tolerancia estará dictada por la frontera de Cramér-Rao.

\section{Objetivo Principal}
\label{sec:obj:main}

Implementar un método de estimación del error incurrido en la localización de fuentes de actividad neuronal al resolver el problema inverso con diferentes valores de la razón de conductividad eléctrica cerebro/cráneo, basado en el cálculo de modelos de electroencefalograma en geometrías realistas obtenidas con el método de elementos de frontera.

\section{Objetivos Particulares}
\label{sec:obj:individual}

\begin{itemize}
	\item Implementar la solución del problema directo en EEG en geometrías realistas con diferentes valores de la razón de conductividad eléctrica cerebro/cráneo, utilizando un dipolo de corriente eléctrica como modelo de un evento de respuesta evocada en distintas zonas de la corteza cerebral representando fuentes de actividad neuronal, y con un nivel de ruido variable.
	\item Calcular el problema inverso en EEG probando los diferentes valores de la razón de conductividad eléctrica cerebro/cráneo en las soluciones del problema directo, y obtener el error incurrido en la localización de fuentes de actividad neuronal entre las utilizadas en la solución del problema directo y las obtenidas en el problema inverso.
	\item Analizar el error incurrido en la localización de fuentes de actividad neuronal utilizando la frontera de Cramér-Rao.
\end{itemize}