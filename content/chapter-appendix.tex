% !TEX root = ../my-thesis.tex
%
\chapter{Anexo}
\label{sec:appendix}

\section{Anexo: comparación	entre los métodos de solución del problema directo en el análisis de fuentes de EEG}
\label{sec:appendix:forward_problem_comparison}


\begin{table}[htb]
	\centering
	\begin{tabular}{@{}lcccc@{}}
	\toprule
											 & BEM       & FEM       & iFDM      & aFDM      \\ \midrule
	Posición de puntos computacionales       & superfice & volumen   & volumen   & volumen   \\
	Elección libre de puntos computacionales & sí        & sí        & no        & no        \\
	Sistemas matriciales                     & completos & dispersos & dispersos & dispersos \\
	Número de compartimentos                 & pequeño   & grande    & grande    & grande    \\
	Requiere teselación                      & sí        & sí        & no        & no        \\
	Anisotrópico                             & no        & sí        & no        & sí        \\ \bottomrule
	\end{tabular}
	\caption{Comparación entre los métodos de solución del problema directo en el análisis de fuentes de EEG. BEM: Boundary Element Method, FEM: Finite Element Method, iFDM: isotropic Finite Difference Method, aFDM: anisotropic Finite Difference Method.}
	\label{tab:forward_problem_comparison}
	\end{table}

\newpage
\section{Anexo: comparación entre los métodos de estimación de la conductividad de los tejidos}
\label{sec:appendix:conductivity_comparison}


\begin{table}[htb]
	\begin{tabular}{@{}lll@{}}
		\toprule
		Método & Fortalezas                                                                                                                                                                       & Limitantes                                                                                                                                                    \\ \midrule
		DAC    & \begin{tabular}[c]{@{}l@{}}- No se requiere un modelo computacional\\ - Posibilidad de clasificar todo tejido\\ - Portabilidad\\ - Asequible\\ - Rápida adquisición\end{tabular} & \begin{tabular}[c]{@{}l@{}}- Invasivo\\ - Condiciones no naturales ex vivo\\ - Homogéneo\end{tabular}                                                         \\
		EIT    & \begin{tabular}[c]{@{}l@{}}- No invasivo\\ - In vivo\\ - Portable\\ - Asequible\\ - Rápida adquisición\end{tabular}                                                              & \begin{tabular}[c]{@{}l@{}}- Requiere modelo computacional\\ - Poca resolución espacial\\ - Baja relación señal/ruido\\ - Homogéneo\end{tabular}              \\
		E/MEG  & \begin{tabular}[c]{@{}l@{}}- No invasivo\\ - In vivo\\ - Portable\\ - Asequible\\ - Rápida adquisición\end{tabular}                                                              & \begin{tabular}[c]{@{}l@{}}- Requiere modelo computacional\\ - Baja resolución espacial\\ - Homogéneo\end{tabular}                                            \\
		MREIT  & \begin{tabular}[c]{@{}l@{}}- No invasivo\\ - In vivo\\ - Alta resolución espacial\\ - Anisotrópico\end{tabular}                                                                  & \begin{tabular}[c]{@{}l@{}}- Baja relación señal/ruido\\ - Señal débil en cráneo\\ - No portable\\ - Relativamente costoso\\ - Lenta adquisición\end{tabular} \\
		DTI    & \begin{tabular}[c]{@{}l@{}}- No invasivo\\ - In vivo\\ - Resolución espacial alta\\ - Anisotrópico\\ - Heterogéneo\end{tabular}                                                  & \begin{tabular}[c]{@{}l@{}}- No portable\\ - Relativamente costoso\\ - Señal débil en cráneo\\ - Lenta adquisición\end{tabular}                               \\ \bottomrule
	\end{tabular}
	\caption{Comparación entre los métodos de estimación de la conductividad de los tejidos. DAC: Dielectric Absorption Capacity, EIT: Electrical Impedance Tomography, E/MEG: Electro/Magnetoencephalography, MREIT: Magnetic Resonance Electrical Impedance Tomography, DTI: Diffusion Tensor Imaging.}
	\label{tab:conductivity_comparison}
\end{table}

