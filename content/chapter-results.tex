% !TEX root = ../my-thesis.tex
%
\chapter{Resultados}
\label{sec:results}

Los resultados obtenidos de la implementación del problema directo y del problema inverso del EEG, así como el análisis estadístico del estimador, se presentan en tres subsecciones: (i) la solución del problema directo, (ii) la solución del problema inverso, y (iii) el análisis estadístico del error incurrido en el uso de diferentes valores del BSCR en la simulación y posterior localización de fuentes de actividad neuronal.
A su vez, debido a la gran cantidad de datos obtenidos, se presentan los de mayor relevancia en las figuras y tablas correspondientes, haciendo especial énfasis en los resultados provenientes del uso de los valores de BSCR más referidos en la literatura y los que más se alejan de estos valores, con el fin de comparar los resultados obtenidos y analizar el desempeño del estimador en cada caso.

\section{Solución del Problema Directo}
\label{sec:results:direct}

Las señales de EEG simuladas obtenidas de la implementación del problema directo se muestran en la \cref{fig:eeg-simulated}. 
Estas señales corresponden a BSCR = 20, 80, 10 y 200, respectivamente, generadas por el dipolo ubicado en la zona somatosensorial y simuladas con un SNR de 1\% proveniente de la fuente.
Esos valores de BSCR fueron seleccionados para comparar los resultados obtenidos en la localización de la fuente de actividad neuronal por su relevancia en la literatura (BSCR = 20 y 80) y por ser los valores más alejados de estos (BSCR = 10 y 200), debido a que presentar los resultados individuales de todos los valores no aportaría grandes diferencias y extendería el documento sin causa.

\begin{figure}[tbp]
    \centering
    \begin{subfigure}{0.9\textwidth}
        \includegraphics[width=\textwidth]{eeg/eeg-d1n1c8c8.png}
        \caption{Señales de EEG simuladas con BSCR = 10.}
        \label{fig:eeg-d1n1c8c8}
        \vspace{0.5em}
    \end{subfigure}
    \vfill
    \begin{subfigure}{0.9\textwidth}
        \includegraphics[width=\textwidth]{eeg/eeg-d1n1c9c9.png}
        \caption{Señales de EEG simuladas con BSCR = 20.}
        \label{fig:eeg-d1n1c9c9}
        \vspace{0.5em}
    \end{subfigure}
    \vfill
    \begin{subfigure}{0.9\textwidth}
        \includegraphics[width=\textwidth]{eeg/eeg-d1n1c10c10.png}
        \caption{Señales de EEG simuladas con BSCR = 80.}
        \label{fig:eeg-d1n1c10c10}
        \vspace{0.5em}
    \end{subfigure}
    \vfill
    \begin{subfigure}{0.9\textwidth}
        \includegraphics[width=\textwidth]{eeg/eeg-d1n1c2c2.png}
        \caption{Señales de EEG simuladas con BSCR = 200.}
        \label{fig:eeg-d1n1c2c2}
    \end{subfigure}
    \caption{Señales de EEG simuladas con diferentes valores de BSCR.}
    \label{fig:eeg-simulated}
\end{figure}

En cada gráfica, se superponen las señales individuales capturadas por cada uno de los 63 electrodos considerados en el análisis, creando un gráfico de mariposa donde el eje horizontal representa el tiempo y el eje vertical representa la amplitud de la señal. 
Este conjunto de señales son una parte del total de las 9000 señales simuladas por las 100 pruebas de cada permutación de BSCR, SNR y fuente de actividad neuronal.

Observando las señales simuladas, se puede apreciar que a medida que el valor de BSCR disminuye, la amplitud de las señales aumenta. 
Esto no indica que la actividad neuronal sea mayor, sino que la señal es más fácil de detectar por la menor resistencia que presenta al pasar por los tejidos del cráneo.
Cabe mencionar que esta diferencia de amplitud no representa una métrica aceptable para comparar la actividad neuronal entre diferentes valores de BSCR, ya que también influyen otros factores como el ruido introducido en la simulación, el ruido generado por el equipo de medición y en casos reales, las condiciones del paciente.
Independientemente del valor de BSCR, se puede observar que las señales presentan un patrón similar, lo que indica que la actividad neuronal simulada es la misma en todos los casos.

\section{Solución del Problema Inverso}
\label{sec:results:inverse}

En la \cref{fig:inverse-results}, se muestran los resultados de la implementación del problema inverso para el valor de BSCR = 10, 20, 80 y 200, correspondientes al conjunto de señales simuladas con un SNR de 1\% proveniente de la fuente de actividad neuronal ubicada en la zona somatosensorial mencionadas en la \cref{sec:results:direct}.
Las señales de EEG simuladas tienen una componente en el tiempo, por tanto, el problema inverso se calcula para cada instante de tiempo correspondiente a la frecuencia de muestreo de 1000 Hz, por lo que los resultados mostrados en la \cref{fig:inverse-results} corresponden al instante de tiempo en el que la función del dipolo utilizando como fuente de actividad neuronal se encuentra en su cenit.

\begin{figure}[tb]
    \centering
    \begin{subfigure}{0.45\textwidth}
        \includegraphics[width=\textwidth]{inverse/inverse-d1n1c2c2.png}
        \caption{PNAI obtenido con BSCR = 200 en problema directo e inverso.}
        \label{fig:inverse-d1n1c2c2}
    \end{subfigure}
    \hfill
    \begin{subfigure}{0.45\textwidth}
        \includegraphics[width=\textwidth]{inverse/inverse-d1n1c8c8.png}
        \caption{PNAI obtenido con BSCR = 10 en problema directo e inverso.}
        \label{fig:inverse-d1n2c2c2}
    \end{subfigure}
    \vskip\baselineskip
    \begin{subfigure}{0.45\textwidth}
        \includegraphics[width=\textwidth]{inverse/inverse-d1n1c9c9.png}
        \caption{PNAI obtenido con BSCR = 20 en problema directo e inverso.}
        \label{fig:inverse-d1n3c2c2}
    \end{subfigure}
    \hfill
    \begin{subfigure}{0.45\textwidth}
        \includegraphics[width=\textwidth]{inverse/inverse-d1n1c10c10.png}
        \caption{PNAI obtenido con BSCR = 80 en problema directo e inverso.}
        \label{fig:inverse-d1n4c2c2}
    \end{subfigure}
    \caption{Resultados de la implementación del problema inverso con diferentes valores de BSCR simulados en el problema directo e inverso en el dipolo ubicado en la zona somatosensorial, la cual se presenta como el área roja \TODO{cambiar color por uno más contrastante}.}
    \label{fig:inverse-results}
\end{figure}

Los resultados mostrados en la figura en cuestión, representan la proyección de la fuente de actividad neuronal en la malla representativa de la corteza cerebral en forma del mapa del PNAI.
Las mallas tienen una área marcada en rojo \TODO{cambair color} que corresponde a la región donde se realizó la búsqueda de las fuentes de actividad neuronal en la simulación de las señales de EEG para el dipolo ubicado en la zona somatosensorial.
Mientras que las áreas entre azul y amarillo correspondientes al rango inscrito en las figuras, representan la magnitud del PNAI, donde el azul indica la menor magnitud y el amarillo la mayor magnitud.

Las figuras muestran la localización de la fuente de actividad neuronal utilizando el mismo valor de BSCR como parámetro del filtro espacial en la solución del problema inverso y en la simulación de las señales de EEG en el problema directo.
Por lo tanto, se espera que el punto de máxima actividad neuronal proyectado sobre la malla representativa de la corteza cerebral se localice en la misma región y vértice que el dipolo utilizado como fuente de actividad neuronal en la simulación de las señales de EEG.
Justo como en los resultados de la \cref{sec:results:direct}, en la \cref{fig:inverse-results} se aprecia una diferencia en la magnitud del PNAI obtenido, donde a medida que el valor de BSCR disminuye, no solo la magnitud del PNAI aumenta, sino que también el área sobre la que se propaga la actividad neuronal simulada se expande.

\section{Estimación del Error en la Localización de la Fuente de Actividad Neuronal}
\label{sec:results:error}

La estimación del error en la localización se obtuvo con el método descrito en la \cref{sec:methodology:estimator}, mientras que el análisis estadístico del desempeño del estimador se realizó con el método descrito en la \cref{sec:methodology:cbr-analysis}.
Por lo tanto, los resultados presentados corresponden a la diferencia entre la localización de la fuente de actividad neuronal obtenida en el problema inverso y la localización de la fuente de actividad neuronal utilizada en la simulación de las señales de EEG en el problema directo dividida por la distancia media entre los vértices de la malla representativa de la corteza cerebral. 
Para que los resultados obtenidos pudieran ser comparados y representados de manera gráfica, se agruparon los valores del error en la localización de la fuente de actividad neuronal obtenidos en las 100 pruebas de cada permutación de BSCR, SNR y fuente de actividad neuronal.



\subsection{Error en el grupo de señales de EEG simuladas con el dipolo en la zona somatosensorial y SNR 1\%}
\label{sec:results:error:d1n1}

En la \cref{fig:error-results-d1n1} se presentan los resultados obtenidos en la estimación del error en la localización de la fuente de actividad neuronal con el dipolo ubicado en la zona somatosensorial (dipolo 1 de acuerdo a nuestro sistema de identificación) y un SNR de 1\%.
Las gráficas representan el error en la localización de la fuente de actividad neuronal para los valores de BSCR = 10, 20, 80 y 200. 






En cada gráfica, el eje horizontal categoriza los valores de BSCR utilizados en la solución del problema inverso y el eje vertical representa el error en la localización de la fuente de actividad neuronal obtenido en las 100 pruebas de cada permutación de BSCR en el problema inverso conforme al método descrito en la \cref{sec:methodology:estimator}. 
epresenta la CRB y el eje vertical representa el error en la localización de la fuente de actividad neuronal obtenido en las 100 pruebas de cada permutación de BSCR en el problema inverso.

\begin{figure}[tb]
    \centering
    \begin{subfigure}{0.49\textwidth}
        \includegraphics[width=\textwidth]{individual_graphs/d1c2n1.png}
        \caption{Error en la localización de la fuente de actividad neuronal con BSCR 200.}
        \label{fig:error-d1c2n1}
    \end{subfigure}
    \hfill
    \begin{subfigure}{0.49\textwidth}
        \includegraphics[width=\textwidth]{individual_graphs/d1c8n1.png}
        \caption{Error en la localización de la fuente de actividad neuronal con BSCR 10.}
        \label{fig:error-d1c8n1}
    \end{subfigure}
    \vskip\baselineskip
    \begin{subfigure}{0.49\textwidth}
        \includegraphics[width=\textwidth]{individual_graphs/d1c9n1.png}
        \caption{Error en la localización de la fuente de actividad neuronal con BSCR 20.}
        \label{fig:error-d1c9n1}
    \end{subfigure}
    \hfill
    \begin{subfigure}{0.49\textwidth}
        \includegraphics[width=\textwidth]{individual_graphs/d1c10n1.png}
        \caption{Error en la localización de la fuente de actividad neuronal con BSCR 80.}
        \label{fig:error-d1c10n1}
    \end{subfigure}
    \caption{Error incurrido en la localización de la fuente de actividad neuronal en con el dipolo en la zona somatosensorial y SNR 1\%.}
    \label{fig:error-results-d1n1}
\end{figure}





Tomando en cuenta los resultados obtenidos y la atipicidad de los valores de BSCR 10 y 200, nos enfocamos en los valores de BSCR 20 y 80, los cuales son los más cercanos a los valores reportados en la literatura y siguen siendo distantes entre sí. 


\subsection{Error en el grupo de señales de EEG simuladas en diferentes zonas de la corteza cerebral y niveles de SNR para BSCR = 20 y 80}
\label{sec:results:error:dn-rest}



Con el fin de ampliar el panorama de los datos obtenidos, se presentan los resultados de la estimación del error en la localización de la fuente de actividad neuronal para los valores de BSCR 20 y BSCR 80.
Específicamente, se comparan los resultados obtenidos de la implementación del problema inverso sobre las señales de EEG simuladas con el dipolo ubicado en las zonas auditiva (dipolo 2) y visual (dipolo 3) y con niveles de SNR: 1\%, 5\% y 10\%.

\begin{figure}[tb]
    \centering
    \begin{subfigure}{0.49\textwidth}
        \includegraphics[width=\textwidth]{individual_graphs/d2c9n1.png}
        \caption{Error en la localización de la fuente de actividad neuronal con BSCR = 20 y SNR = 1\%.}
        \label{fig:error-d2c9n1}
    \end{subfigure}
    \hfill
    \begin{subfigure}{0.49\textwidth}
        \includegraphics[width=\textwidth]{individual_graphs/d2c10n1.png}
        \caption{Error en la localización de la fuente de actividad neuronal con BSCR = 80 y SNR = 1\%.}
        \label{fig:error-d2c10n1}
    \end{subfigure}
    \vskip\baselineskip
    \begin{subfigure}{0.49\textwidth}
        \includegraphics[width=\textwidth]{individual_graphs/d2c9n2.png}
        \caption{Error en la localización de la fuente de actividad neuronal con BSCR = 20 SNR = 5\%.}
        \label{fig:error-d2c9n2}
    \end{subfigure}
    \hfill
    \begin{subfigure}{0.49\textwidth}
        \includegraphics[width=\textwidth]{individual_graphs/d2c10n2.png}
        \caption{Error en la localización de la fuente de actividad neuronal con BSCR = 80 y SNR = 5\%.}
        \label{fig:error-d2c10n2}
    \end{subfigure}
    \vskip\baselineskip
    \begin{subfigure}{0.49\textwidth}
        \includegraphics[width=\textwidth]{individual_graphs/d2c9n3.png}
        \caption{Error en la localización de la fuente de actividad neuronal con BSCR = 20 y SNR = 10\%.}
        \label{fig:error-d2c9n3}
    \end{subfigure}
    \hfill
    \begin{subfigure}{0.49\textwidth}
        \includegraphics[width=\textwidth]{individual_graphs/d2c10n3.png}
        \caption{Error en la localización de la fuente de actividad neuronal con BSCR = 80 y SNR = 10\%.}
        \label{fig:error-d2c10n3}
    \end{subfigure}
    \caption{Error incurrido en la localización de la fuente de actividad neuronal en con el dipolo en la zona auditiva y los tres niveles de SNR.}
    \label{fig:error-results-d2}
\end{figure}




\begin{figure}[tb]
    \centering
    \begin{subfigure}{0.49\textwidth}
        \includegraphics[width=\textwidth]{individual_graphs/d3c9n1.png}
        \caption{Error en la localización de la fuente de actividad neuronal con BSCR = 20 y SNR = 1\%.}
        \label{fig:error-d3c9n1}
    \end{subfigure}
    \hfill
    \begin{subfigure}{0.49\textwidth}
        \includegraphics[width=\textwidth]{individual_graphs/d3c10n1.png}
        \caption{Error en la localización de la fuente de actividad neuronal con BSCR = 80 y SNR = 1\%.}
        \label{fig:error-d3c10n1}
    \end{subfigure}
    \vskip\baselineskip
    \begin{subfigure}{0.49\textwidth}
        \includegraphics[width=\textwidth]{individual_graphs/d3c9n2.png}
        \caption{Error en la localización de la fuente de actividad neuronal con BSCR = 20 SNR = 5\%.}
        \label{fig:error-d3c9n2}
    \end{subfigure}
    \hfill
    \begin{subfigure}{0.49\textwidth}
        \includegraphics[width=\textwidth]{individual_graphs/d3c10n2.png}
        \caption{Error en la localización de la fuente de actividad neuronal con BSCR = 80 y SNR = 5\%.}
        \label{fig:error-d3c10n2}
    \end{subfigure}
    \vskip\baselineskip
    \begin{subfigure}{0.49\textwidth}
        \includegraphics[width=\textwidth]{individual_graphs/d3c9n3.png}
        \caption{Error en la localización de la fuente de actividad neuronal con BSCR = 20 y SNR = 10\%.}
        \label{fig:error-d3c9n3}
    \end{subfigure}
    \hfill
    \begin{subfigure}{0.49\textwidth}
        \includegraphics[width=\textwidth]{individual_graphs/d3c10n3.png}
        \caption{Error en la localización de la fuente de actividad neuronal con BSCR = 80 y SNR = 10\%.}
        \label{fig:error-d3c10n3}
    \end{subfigure}
    \caption{Error incurrido en la localización de la fuente de actividad neuronal en con el dipolo en la zona auditiva y los tres niveles de SNR.}
    \label{fig:error-results-d3}
\end{figure}



\section{Desempeño del Error Incurrido en la Localización de Fuentes de Actividad Neuronal}
\label{sec:results:bscr-performance}

\begin{figure}[tb]
    \centering
    \begin{subfigure}{0.49\textwidth}
        \includegraphics[width=\textwidth]{overall_bscr_error_boxplots_percentile/d1n1.png}
        \caption{Error en la localización de la fuente de actividad neuronal de diferentes BSCR: Dipolo 1, SNR 1\%.}
        \label{fig:error-overall-d1n1}
    \end{subfigure}
    \hfill
    \begin{subfigure}{0.49\textwidth}
        \includegraphics[width=\textwidth]{overall_bscr_error_boxplots_percentile/d1n3.png}
        \caption{Error en la localización de la fuente de actividad neuronal de diferentes BSCR: Dipolo 1, SNR 10\%.}
        \label{fig:crb-overall-d1n3}
    \end{subfigure}
    \vskip\baselineskip
    \begin{subfigure}{0.49\textwidth}
        \includegraphics[width=\textwidth]{overall_bscr_error_boxplots_percentile/d2n1.png}
        \caption{Error en la localización de la fuente de actividad neuronal de diferentes BSCR: Dipolo 2, SNR 1\%.}
        \label{fig:error-overall-d2n1}
    \end{subfigure}
    \hfill
    \begin{subfigure}{0.49\textwidth}
        \includegraphics[width=\textwidth]{overall_bscr_error_boxplots_percentile/d2n3.png}
        \caption{Error en la localización de la fuente de actividad neuronal de diferentes BSCR: Dipolo 2, SNR 10\%.}
        \label{fig:crb-overall-d2n3}
    \end{subfigure}
    \vskip\baselineskip
    \begin{subfigure}{0.49\textwidth}
        \includegraphics[width=\textwidth]{overall_bscr_error_boxplots_percentile/d3n1.png}
        \caption{Error en la localización de la fuente de actividad neuronal de diferentes BSCR: Dipolo 3, SNR 1\%.}
        \label{fig:error-overall-d3n1}
    \end{subfigure}
    \hfill
    \begin{subfigure}{0.49\textwidth}
        \includegraphics[width=\textwidth]{overall_bscr_error_boxplots_percentile/d3n3.png}
        \caption{Error en la localización de la fuente de actividad neuronal de diferentes BSCR: Dipolo 3, SNR 10\%.}
        \label{fig:crb-overall-d3n3}
    \end{subfigure}
    \caption{Desempeño del uso de diferentes valores de BSCR en la simulación y localización de fuentes de actividad neuronal.}
    \label{fig:bscr-performance}
\end{figure}

En la \cref{fig:bscr-performance}, se presentan los resultados obtenidos en la estimación del error en la localización de la fuente de actividad neuronal para cada uno de los BSCR involucrados en el estudio, agrupados por el dipolo y nivel de ruido correspondiente.
A detalle, el eje horizontal de cada gráfica representa los valores de BSCR utilizados en la simulación de las señales de EEG y el eje vertical representa el error general en la localización de la fuente de actividad neuronal obtenido en las 100 pruebas de cada permutación de BSCR en el problema inverso, por lo tanto, se trata de una matriz de 1000$\times$10.
Se eligió mostrar los resultados del nivel de SNR 1\% y 10\% para cada dipolo, ya que estos representan los extremos en la relación señal-ruido y permiten observar el desempeño del estimador en condiciones de ruido bajo y ruido alto.

